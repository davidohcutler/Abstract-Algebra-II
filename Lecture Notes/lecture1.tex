\documentclass[12pt]{article}

\usepackage[margin=1in]{geometry} 
\usepackage{amsmath,amsthm,amssymb,enumitem}

\newcommand{\N}{\mathbb{N}}
\newcommand{\Z}{\mathbb{Z}}
\newcommand{\R}{\mathbb{R}}
\newcommand{\Rd}{\mathbb{R}^{d}}
\newcommand{\exr}{[-\infty, \infty]}

\newenvironment{ex}[2][Exercise]{\begin{trivlist}
\item[\hskip \labelsep {\bfseries #1}\hskip \labelsep {\bfseries #2.}]}{\end{trivlist}}

\newenvironment{sol}[1][Solution]{\begin{trivlist}
\item[\hskip \labelsep {\bfseries #1:}]}{\end{trivlist}}

\newenvironment{theorem}[2][Theorem]{\begin{trivlist}
    \item[\hskip \labelsep {\bfseries #1}\hskip \labelsep {\bfseries #2.}]}{\end{trivlist}}
    
\newenvironment{lemma}[2][Lemma]{\begin{trivlist}
    \item[\hskip \labelsep {\bfseries #1}\hskip \labelsep {\bfseries #2.}]}{\end{trivlist}}
    
\newenvironment{definition}[2][Definition]{\begin{trivlist}
    \item[\hskip \labelsep {\bfseries #1}\hskip \labelsep {\bfseries #2.}]}{\end{trivlist}}
    
\newenvironment{example}[1][Example]{\begin{trivlist}
    \item[\hskip \labelsep {\bfseries #1:}]}{\end{trivlist}}


\begin{document}
\noindent David O.H. Cutler \hfill {\Large Math 146: Lecture 1} \hfill \today

\begin{example}
    Why is $S_3$ "built from abelian groups"? \\ \\
    Recall: the order of $S_3$ is $6$. $S_3$ has a normal subgroup $C_3 \subseteq S_3$, where $|C_3| = 3$. \\ \\
    Because it is normal, we can construct the quotient group $S_3 / C_3$ which has order $2$. This is one group up to isomorphism of order 2, so it is abelian given we must have $S_3 / C_3 \cong C_2$. \\ \\
    So $S_3$ has a normal, abelian subgroup that has the property that the quotient of $S_3$ with it is also abelian; we appeal to this for intuition as to what "built from abelians means". \\ \\
    Continuing this example on $S_4$, we note that $A_4$ does not work, as it isn't abelian. Considering over the Klein group however, we get interestingly that $S_4 / V_4 \cong S_3$. \\ \\
    So we extend our definition that $S_4$ is "built from abelians" in the sense that it has an abelian normal subgroup which determines a quotient that itself is "built from abelians".
\end{example}

\begin{definition}{(Solvable)}
    A group which is "built from abelians" is also called \textbf{solvable}.
\end{definition}

\begin{example}
    This is why the quintic formula fails. For $n \geq 5$, the only normal subgroups of $S_n$ are the trivial subgroup and $A_n$, but $A_n$ for $n \geq 5$ is simple and thus has no meaningful quotient.
\end{example}

\begin{definition}{(Ring)}
    A \textbf{ring} is an abelian group $R$ (with operation $+$) with an additional operation (denoted $\cdot$) such that:
    \begin{enumerate}[label=(\alph*)]
        \item Multiplication is associative.
        \item Multiplication distributes over addition both ways.
        \item There is a multiplicative identity. 
    \end{enumerate}
\end{definition}

\begin{lemma}
    e = ee' = e' 

    0 multiply by anything
\end{lemma}





\end{document}