\documentclass[12pt]{article}

\usepackage[margin=1in]{geometry} 
\usepackage{amsmath,amsthm,amssymb,enumitem,bbm,xfrac}

\newcommand{\N}{\mathbb{N}}
\newcommand{\Z}{\mathbb{Z}}
\newcommand{\R}{\mathbb{R}}
\newcommand{\Rd}{\mathbb{R}^{d}}
\newcommand{\exr}{[-\infty, \infty]}

\newenvironment{ex}[2][Exercise]{\begin{trivlist}
\item[\hskip \labelsep {\bfseries #1}\hskip \labelsep {\bfseries #2.}]}{\end{trivlist}}

\newenvironment{sol}[1][Solution]{\begin{trivlist}
\item[\hskip \labelsep {\bfseries #1:}]}{\end{trivlist}}

\newenvironment{theorem}[2][Theorem]{\begin{trivlist}
\item[\hskip \labelsep {\bfseries #1}\hskip \labelsep {\bfseries #2.}]}{\end{trivlist}}
    
\newenvironment{lemma}[2][Lemma]{\begin{trivlist}
\item[\hskip \labelsep {\bfseries #1}\hskip \labelsep {\bfseries #2.}]}{\end{trivlist}}
    
\newenvironment{definition}[2][Definition]{\begin{trivlist}
\item[\hskip \labelsep {\bfseries #1}\hskip \labelsep {\bfseries #2.}]}{\end{trivlist}}
    
\newenvironment{example}[1][Example]{\begin{trivlist}
\item[\hskip \labelsep {\bfseries #1:}]}{\end{trivlist}}
\begin{document}
\noindent David Owen Horace Cutler \hfill {\Large Math 146: Homework 3} \hfill \today

\begin{ex}{1}
    Let $F$ be a field and let $f \in F[x]$ have degree $1$. Prove that the principal ideal generated by $f$ is maximal.
    \begin{proof}
        Assume for the sake of contradiction that $(f)$ is not maximal, i.e. there is some ideal $I$ such that:
        $$(f) \subsetneq I \subsetneq F[x]$$
        Obviously then, $I \setminus (f)$ is nonempty then. We can consider some polynomial $g$ in $I \setminus (f)$ then. \\ \\
        By definition, $f \nmid g$, as this would have $g \in (f)$. Appealing to the division algorithm for polynomials then, we have a unique decomposition of $g$ given by the following:
            \begin{equation}
                \begin{aligned}
                g = fq + r \\
                q, r \in F[x], \; \text{deg}(r) < \text{deg}(f)
                \end{aligned}
            \end{equation}
            We note it cannot be $r = 0$, as this would have $g = fq$, i.e. $f \mid g$. As $\text{deg}(r) < \text{deg}(f) = 1$ then, the only possibility is that $\text{deg}(r) = 0$, i.e. $r$ is a constant polynomial. \\ \\
            As $f \nmid g$ then, it must be $r \neq 0$. Reorganizing, we get
            $$r = g - fq,$$
            where $r \in I$ as $g \in I$ and $-fq = f(-q) \in (f) \subsetneq I$, appealing to fact that $I$ as an ideal is additive subgroup. \\ \\
            Thus $I$ contains a constant polynomial $r = a \neq 0 \in F$. As $a$ is a unit in $F$ then, it follows $r$ is a unit in $F[x]$, i.e. it admits an inverse $r^{-1} \in F[x]$. \\ \\
            As $r \in I$ then, $rr^{-1} = 1 \in I$. But this yields a contradiction as it follows then trivially that $I$ must be the whole ring $F[x]$, as $1 \in I$ has that $1 \cdot h = h \in I$ for all $h \in F[x]$. \\ \\
            Thus it must be the principal ideal $(f)$ is maximal.
    \end{proof}
\end{ex}

\begin{ex}{2}
    Let $R$ be a commutative ring. Prove that every maximal ideal is a prime ideal.
    \begin{proof}
        Using the theorems we proved in class, we know for a commutative ring $R$ that given an ideal $I$, $\sfrac{R}{I}$ is an integral domain if and only if $I$ is prime, and $\sfrac{R}{I}$ is a field if and only if $I$ is maximal. \\ \\
        Let $I$ be a maximal ideal in $R$ then. Recalling then that all fields are integral domains (as they admit cancellation), we get the following:
        \begin{equation}
            \begin{aligned}
                I \text{ is maximal} 
                \rightarrow \sfrac{R}{I} \text{ is a field} \\
                \rightarrow \sfrac{R}{I} \text{ is an integral domain} 
                \rightarrow I \text{ is prime}
            \end{aligned}
        \end{equation}
        Which shows exactly what is desired.
    \end{proof}
\end{ex}
\begin{ex}{3}
    Let $R = \mathbb{Z}[i]$. For a prime number $p$, let $(p)$ be the principal ideal generated by $p$ in $R$.
    \begin{enumerate}[label=(\alph*)]
        \item Prove that the ideal $(5)$ is not a prime ideal.
        \begin{proof}
            To show $(5)$ is not prime, we will demonstrate that there are some $x, y \in \mathbb{Z}[i]$ such that $xy \in (5)$ but $x \notin (5)$ and $y \notin (5)$.
            \\ \\ For this note we have the following product:
            \begin{equation}
                (2 + i)(2 - i) = 5
            \end{equation}
            So it follows $(2 + i)(2 - i) = 5$. However, we establish neither $2 + i$ nor $2 + i$ in $(5)$ as we have the following:
            \begin{equation}
                \begin{aligned}
                    2 + i = 5x \Leftrightarrow \frac{2}{5} + \frac{i}{5} = x \\
                    2 - i = 5x \Leftrightarrow \frac{2}{5} - \frac{i}{5} = x
                \end{aligned}
            \end{equation}
            But neither $\frac{2}{5} + \frac{i}{5}$ nor $\frac{2}{5} - \frac{i}{5}$ lies in $\mathbb{Z}[i]$. In particular, we note we can formalize this notion by viewing these elements as laying in the field $\mathbb{Q}[i]$. \\ \\
            Specifically, we can appeal to the underlying multiplicative group structure of $\mathbb{Q}[i]$ (sans zero), which has that there are unique $x$ that satisfy:
            \begin{equation}
                \begin{aligned}
                    2 + i = 5x \\
                    2 - i = 5x
                \end{aligned}
            \end{equation}
            Specifically, we note the unique solutions are $\frac{2}{5} + \frac{i}{5}$ and $\frac{2}{5} - \frac{i}{5}$ respectively. \\ \\ Consequently there can be no solution to either laying in $\mathbb{Z}[i]$, as such a solution would also be a solution in $\mathbb{Q}[i]$, in which the unique solutions are explicitly not Gaussian integers. \\ \\
            Thus neither $2 + i$ nor $2 - i$ can be written as a multiple of $5$, and so neither $2 + i$ nor $2 + i$ are in $(5)$. It follows $(5)$ is not prime.
        \end{proof}
        \item Find an ideal $I$ other than $R$ and $(5)$ such that $(5) \subseteq I$. 
        \begin{proof}
            We claim $I = (2 + i)$ suffices. Obviously, by the previous part we have $5 \in (2 + i)$, which suffices to show that $(5) \subseteq (2 + i)$. \\ \\
            Moreover, it is clear that $(2 + i) \neq (5)$, as $2 + i \in (2 + i)$ but $2 + i \notin (5)$, again by the previous part. \\ \\
            Lastly, to show $(2 + i) \neq R$, we just note that $1 \notin (2 + i)$ as $(2 + i)$ is not unit in $\mathbb{Z}[i]$, in particular we can do the same argument in $\mathbb{Q}[i]$:
            \begin{equation}
                (2 + i)^{-1} = \frac{2}{5} - \frac{i}{5}
            \end{equation}
            But the latter term is not in $\mathbb{Z}[i]$, so there is no term $x \in \mathbb{Z}[i]$ such that we have $(2 + i)x = 1$, as seen by appealing the exact same argument as the previous part. \\ \\
            Thus $(2 + i)$ fulfills the desired properties.
        \end{proof}
        \item More generally, suppose $p$ is such that there is a solution to the equation $x^2 + 1 = 0$ in $\mathbb{Z}_p$. Prove that the principal ideal $(p)$ is not prime.
        \begin{proof}
            Let $[\alpha]_p$ be a solution to $x^2 + 1$ in $\mathbb{Z}_p$, i.e.
            \begin{equation}
                [\alpha]_p^2 + [1]_p = [0]_p,
            \end{equation}
            it t follows $\alpha^2 + 1 \equiv 0 \pmod{p}$, i.e. $\alpha^2 + 1 = pq$ for some $q \in \mathbb{Z}$. Factoring yields
            \begin{equation}
                (\alpha + i)(\alpha - i) = pq
            \end{equation}
            And so $(\alpha + i)(\alpha - i) \in (p)$. However, we note neither $\alpha + i \in (p)$ nor $\alpha - i \in (p)$ as working over $\mathbb{Q}[i]$ we have
            \begin{equation}
                \begin{aligned}
                    \alpha + i = px \Leftrightarrow \frac{\alpha}{p} + \frac{i}{p} = x \\
                    \alpha - i = px \Leftrightarrow \frac{\alpha}{p} - \frac{i}{p} = x
                \end{aligned}
            \end{equation}
            and so appealing to the exact same argument as the previous two parts then, in particular as $\frac{1}{p} \notin \mathbb{Z}$, we note there can be no Gaussian integer solutions to $x$. \\ \\
            Thus neither $\alpha - i \in (p)$ nor $\alpha + i \in (p)$. Thus $(p)$ is not prime.
        \end{proof}
        \item On the other hand, prove that $(3)$ \textit{is} maximal.
        \begin{proof}
            I'm not sure there is a more streamlined way to do this then appealing to the argument in (e), so I just verify that there is no solution to $x^2 + 1 = 0$ in $\mathbb{Z}_3$. \\ \\
            In particular, we note:
            \begin{equation}
                \begin{aligned}
                    0^2 + 1 \equiv 1 \pmod{3} \\
                    1^2 + 1 \equiv 2 \pmod{3} \\
                    2^2 + 1 \equiv 2 \pmod{3}
                \end{aligned}
            \end{equation}
            Using (e) then, we note $(3)$ in maximal in $\mathbb{Z}[i]$.
        \end{proof}
        \item Suppose that $p$ is any prime such that there is no solution to the equation $x^2 + 1 = 0$ in $\mathbb{Z}_p$. Prove that the ideal $(3)$ is maximal.
        \begin{proof}
            Assume for the sake of contradiction $(p)$ is not maximal, i.e.
            \begin{equation}
                (p) \subsetneq I \subsetneq R
            \end{equation}
            For some ideal $I$. By assumption then, we can consider can consider some $a + bi \in I \setminus (p)$. \\ \\
            As $a + bi \notin (p)$, it must be that $p \nmid a$ or $p \nmid b$. Note then that as $x^2 + 1$ has no solutions over $\mathbb{Z}_p$ and is quadratic, it must be irreducible over $\mathbb{Z}_p[x]$. \\ \\
            In particular, if it were not irreducible, it would admit a factorization into linear polynomials which would assuredly admit solutions, contradicting that $x^2 + 1$ has no roots. \\ \\
            As $\mathbb{Z}_p$ is a field then, appealing to a result proven in class, we have that $\sfrac{\mathbb{Z}_p[x]}{(x^2 + 1)}$ is a field, given the irreducibility of $x^2 - 1$. \\ \\
            As $\sfrac{\mathbb{Z}_p[x]}{(x^2 + 1)}$ is ``essentially" $\mathbb{Z}_p[x]$ collapsed onto $x^2 + 1$, we can view it as $\mathbb{Z}_p[x]$ with the additional rule that $x^2 = -1$. Recalling then that $p \nmid a$ or $p \nmid b$, we have
            \begin{equation}
                [a]_p + [b]_px \neq 0
            \end{equation}
            As $\mathbb{Z}_p[x]$ with $x^2 = -1$ is a field then, $[a]_p + [b]_px$ is a unit here, i.e. it admits an inverse $[a']_p + [b']_px$. Thus
            \begin{equation}
                [a]_p + [b]_px([a']_p + [b']_px) = [aa' - bb'] + [ab' + ba']_px = [1]_p
            \end{equation}
            In particular then, it must be $[ab' + ba']_p = [0]_p$ and thus $[aa' - bb']_p = [1]_p$. \\ \\
            This is specifically that $ab' + ba' = mp$ and $aa' - bb' = 1 + kp$ for $m, k \in \mathbb{Z}$. Back in $\mathbb{Z}[i]$ now, consider the multiplication
            \begin{equation}
                (a + bi)(a' + b'i) = (aa' - bb') + (ab' + ba')i
            \end{equation}
            , where this product is in $I$ given $a + bi \in I$. It follows
            \begin{equation}
                (a + bi)(a' + b'i) = (aa' - bb') + (ab' + ba')i = 1 + kp + mpi = 1 + p(k + mi) \in I
            \end{equation}
            But obviously then $kp + mpi = p(k + mi) \in (p) \subsetneq I$, so using the properties of the ideal we have
            \begin{equation}
                1 + p(k + mi) - p(k + mi) = 1 \in I
            \end{equation}
            But this is a contradiction as then clearly $I$ becomes the whole ring $R$, but we assumed $I \subsetneq R$. Thus $(p)$ must be maximal.
        \end{proof}
    \end{enumerate}
\end{ex}

\begin{ex}{4}
    Let $f \in \mathbb{Q}[x]$ given by $f = x^3 - 2$.
    \begin{enumerate}[label=(\alph*)]
        \item Prove that $f$ is irreducible, i.e. that there no do not exist polynomials $f_1, f_2 \in \mathbb{Q}[x]$ with $\text{deg}f_1 = 1$ and $\text{deg}f_2 = 2$ such that $f = f_1f_2$.
        \begin{proof}
            Assume for the sake of contradiction $f$ is not irreducible, i.e. it admits a factorization over $\mathbb{Q}[x]$ given by 
            \begin{equation}
                (ax^2 + bx + c)(dx + e)
            \end{equation}
            Where $a, b, c, d, e \in \mathbb{Q}$ with at least $a, d \neq 0$. Clearly $dx + e$ admits a root $\frac{-e}{d}$ as 
            \begin{equation}
                d\Big (\frac{-e}{d} \Big ) + e = -e + e = 0
            \end{equation}
            So clearly then $\frac{-e}{d}$ is also a root of $f$, i.e. 
            \begin{equation}
                \Big ( \frac{-e}{d} \Big)^3 = 2
            \end{equation}
            Note here it must be $\frac{-e}{d}$ is positive as $x^3$ preserves sign. Taking the (principal) cube root the has $\frac{-e}{d} = \sqrt[3]{2}$, but this is a contradiction as $\sqrt[3]{2}$ is irrational and $\frac{-e}{d} \in \mathbb{Q}$. \\ \\
            It follows $f$ must be irreducible.
        \end{proof}
        \item Conclude that $\sfrac{\mathbb{Q}[x]}{(f)}$ is a field.
        \begin{proof}
            This follows immediately from a theorem we proved in class. In particular, as $\mathbb{Q}$ is a field and $f$ is non-constant irreducible in $\mathbb{Q}[x]$, it follows this is a field.
        \end{proof}
    \item Use the polynomial division algorithm to find polynomials $q, r \in \mathbb{Q}[x]$ such that $x^4 + x + 1 = qf + r$, when $\text{deg}(r) < \text{deg}(f)$. Observe that $r(\sqrt[3]{2}) = \sqrt[3]{2}^4 + \sqrt[3]{2} + 1$.
    \begin{proof}
        We perform the computtion utilizing the division algorithm:
        \begin{equation}
            \begin{array}{r}
                x\phantom{)}   \\
                x^3-2{\overline{\smash{\big)}\,x^4+x+1\phantom{)}}}\\
                \underline{-~\phantom{(}(x^4-2x)\phantom{-b)}}\\
                3x + 1\phantom{)}
                \end{array}
        \end{equation}
        Thus $x^4 + x + 1 = (x)(x^3 - 2) + (3x + 1)$. Here $q = x$ and $r = 3x + 1$, where we of course have $\text{deg}(r) < \text{deg}(f)$. We note then:
        \begin{equation}
            r(\sqrt[3]{2}) = 3(\sqrt[3]{2}) + 1
        \end{equation}
        We want to show $r(\sqrt[3]{2}) = \sqrt[3]{2}^4 + \sqrt[3]{2} + 1$, so we would like $\sqrt[3]{2}^4 + \sqrt[3]{2} = 3(\sqrt[3]{2})$. For this, just note 
        \begin{equation}
            2\sqrt[3]{2} = \sqrt[3]{2}^4 \Leftrightarrow 2^12^{\frac{1}{3}} = 2^\frac{4}{3} \Leftrightarrow 2^\frac{4}{3} = 2^\frac{4}{3}
        \end{equation}
        Which verifies (21).
    \end{proof}
    \item Use the polynomial Euclidean algorithm to find explicit polynomials $A, B \in \mathbb{Q}[x]$ such that $A \cdot (3x + 1) + B \cdot f = 1$.
    \begin{proof}
        We first apply the division algorithm to $x^3 -2$ and $3x + 1$, which has
        \begin{equation}
            x^3 - 2 = \Big ( \frac{x^2}{3} - \frac{x}{9} + \frac{1}{27} \Big)(3x + 1) + \Big (\frac{-55}{27} \Big),
        \end{equation}
        and so we can rearrange to get 
        \begin{equation}
            \frac{55}{27} = \Big ( \frac{x^2}{3} - \frac{x}{9} + \frac{1}{27} \Big)(3x + 1) - (x^3 - 2),
        \end{equation}
        which finally has
        \begin{equation}
            1 = \frac{27}{55} \Big ( \frac{x^2}{3} - \frac{x}{9} + \frac{1}{27} \Big)(3x + 1) - \frac{27}{55}(x^3 - 2)
        \end{equation}
        Expanding then, we can get the desired polynomials are $A = \frac{9x^2}{55} - \frac{3x}{55} + \frac{1}{55}$ and $B = -\frac{27}{55}$.
    \end{proof}
    \item Express $\frac{1}{3\sqrt[3]{2} + 1}$ in the form $a + b\sqrt[3]{2} + c\sqrt[3]{2}$, for $a, b, c \in \mathbb{Q}$.
    \begin{proof}
        This is just a routine application of the last part, plugging into $A$. \\ \\
        Doing this yields:
        \begin{equation}
            \frac{1}{3\sqrt[3]{2} + 1} = \frac{9}{55}\sqrt[3]{4} - \frac{3}{55} \sqrt[3]{2} + \frac{1}{55}
        \end{equation}
        Which is exactly what is desired.
    \end{proof}
    \end{enumerate}
\end{ex}

\end{document}