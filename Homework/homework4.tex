\documentclass[12pt]{article}

\usepackage[margin=1in]{geometry} 
\usepackage{amsmath,amsthm,amssymb,enumitem,bbm,xfrac}

\newcommand{\N}{\mathbb{N}}
\newcommand{\Z}{\mathbb{Z}}
\newcommand{\R}{\mathbb{R}}
\newcommand{\Rd}{\mathbb{R}^{d}}
\newcommand{\exr}{[-\infty, \infty]}

\newenvironment{ex}[2][Exercise]{\begin{trivlist}
\item[\hskip \labelsep {\bfseries #1}\hskip \labelsep {\bfseries #2.}]}{\end{trivlist}}

\newenvironment{sol}[1][Solution]{\begin{trivlist}
\item[\hskip \labelsep {\bfseries #1:}]}{\end{trivlist}}

\newenvironment{theorem}[2][Theorem]{\begin{trivlist}
\item[\hskip \labelsep {\bfseries #1}\hskip \labelsep {\bfseries #2.}]}{\end{trivlist}}
    
\newenvironment{lemma}[2][Lemma]{\begin{trivlist}
\item[\hskip \labelsep {\bfseries #1}\hskip \labelsep {\bfseries #2.}]}{\end{trivlist}}
    
\newenvironment{definition}[2][Definition]{\begin{trivlist}
\item[\hskip \labelsep {\bfseries #1}\hskip \labelsep {\bfseries #2.}]}{\end{trivlist}}
    
\newenvironment{example}[1][Example]{\begin{trivlist}
\item[\hskip \labelsep {\bfseries #1:}]}{\end{trivlist}}
\begin{document}
\noindent David Owen Horace Cutler \hfill {\Large Math 146: Homework 4} \hfill \today

\begin{ex}{1}
    Let $F$ be a field and let $f \in F[x]$ be a monic polynomial. Prove that $f$ is irreducible if and only if $f$ cannot be factored as a product of two $\textit{monic}$ polynomials of smaller degree.
    \begin{proof}
        Remark the forward direction is trivially true, as irreducibility specifically means that $f$ cannot be factored into \textbf{any} two polynomials of smaller degree.
        \\ \\
        We will prove the reverse direction then, that is that if $f$ cannot be factored as a product of two monic polynomials of smaller degree, then $f$ is irreducible.  \\ \\
        For this, we will consider the contrapositive that is that $f$ being reducible implies that $f$ can be factored as a product of two monic polynomials of smaller degree. \\ \\
        We first remark that as $f$ is reducible, it admits a factorization as follows:
        \begin{equation}
            \begin{aligned}
            f = gh \\
            \text{deg}(g), \; \text{deg}(h) < \text{deg}(f)
            \end{aligned}
        \end{equation}
        In particular, we may write $g = \sum_{i = 0}^N a_ix^i$ and $h = \sum_{j = 0}^M b_jx^j$ where $N, M < \text{deg}(f)$ and $a_N, b_M \neq 0$. We recall then the product formula for $gh$, that is:
        \begin{equation}
            gh = \sum_{l = 0}^{N + M} \Big (\sum_{\substack{i + j = l \\ i, j \in \mathbb{N}}} a_ib_j \Big)x^l  \\
        \end{equation}
        Taking $l = N+M$ then demonstrates that the $N + M$ degree coefficient of $gh = f$ is $a_Nb_M$. As $f$ is monic then, it must be than $a_Nb_M = b_Ma_N = 1$, i.e. $a_N$ and $b_M$ are multiplicative inverses of each other. 
        \\ \\ Using that $F[x]$ is a commutative ring then, we get the following:
        \begin{equation}
            f = gh = (1)(gh) = (a_Nb_M)(gh) = (b_Mg)(a_Nh)
        \end{equation}
        But then clearly the polynomials $b_Mg$ and $a_Nh$ are monic, as the coefficients of the leading terms are $b_Ma_N$ and $a_Nb_M$ respectively, both of which are equal to 1. \\ \\
        Of course, this multiplication does not alter the degree, so we have produced a factorization of $f$ into monic polynomials of strictly lesser degree. \\ \\
        This proves the reverse direction, so the statement holds in general.
    \end{proof}
\end{ex}

\begin{ex}{2}
    Let $p$ be a prime and let $\mathbb{F}_p$ be the field with $p$ elements.
    \begin{enumerate}[label=(\alph*)]
        \item List all possible ways that a monic polynomial $f \in \mathbb{F}_p[x]$ with $\text{deg}(f) = 3$ could factor.
        \begin{proof}
            Ostensibly, there is one possible factorization, that is into a quadratic and linear polynomial, i.e. $f = f_1f_2$ where $f_1$ is degree 2 and $f_2$ is degree $1$. \\ \\
            However, there is a bit more nuance, as if/how the quadratic factors is important to the counting argument. \\ \\
            For this, first note that we can assume every factorization is monic (i.e. $f_1, f_2$ monic) without loss of generality by problem 1. With this in mind, we get 4 interesting possibilites:
            \begin{enumerate}[label=(\roman*)]
                \item $f_1$ is irreducible \\
                \item $f_1$ factors into two linear polynomials with roots distinct from each other and from the root of $f_2$
                \item $f_1$ factors into two linear polynomials such that out of the now three linear polynomials in the decomposition, exactly two share the same root (i.e. multiplicity 2)
                \item $f_1$ factors into linear polynomials such that the now three polynomials in the decomposition all have the same root
            \end{enumerate}
            It is intuitively clear this describes all possibilities, as the only thing which can differ in the reducible cases is how many roots the linear polynomials in the decomposition share. 
        \end{proof}
        \item Prove that there are exactly $\frac{p^3 - p}{3}$ monic, irreducible polynomials $f \in \mathbb{F}_p[x]$ with degree $3$.
        \begin{proof}
            There are clearly $p^3$ possible monic degree 3 polynomials in $\mathbb{F}_p[x]$. \\ \\To find the amount of monic irreducible polynomials then, we just find the amount of monic reducible degree 3 polynomials, which can be done by counting the possible factorizations as outlined in (a). \\ \\
            Say then monic $f$ is reducible and thus we have $f = f_1f_2$ for degree 2 $f_1$ and degree 1 $f_2$. By problem 1, we can assume the factorization is monic.
            \begin{enumerate}[label=(\roman*)]
                \item If $f_1$ is irreducible, using the hint there are $\frac{p^2 - p}{2}$ choices for $f_1$ and clearly $p$ choices for $f_2$ so we get
                $$\Big ( \frac{p^2 - p}{2} \Big)p = \frac{p^3 - p^2}{2}$$
                total choices in this case.
                \item If $f_1$ factors into monic linear polynomials with distinct roots, we seemingly have $p(p - 1)(p - 2)$ choices, but as multiplication commutes we divide this by $3!$, so we get
                $$\frac{p(p-1)(p-2)}{6}$$
                choices in this case.
                \item If $f_1$ factors into monic linear polynomials where exactly two share a root, we seemingly have $p(p-1)$ choices. \\ \\This ends up being true, as we don't need to divide here given we can write all decompositions of this form in the order of 
                $$(x-a)(x-a)(x-b)$$
                where commuting this \textbf{does not} yield something of the same form (unlike the previous part). We thus get the amount of choices from the $p$ choices for $a$ and the $p-1$ choices for $b$, giving
                $$p(p-1)$$
                choices in this case.
                \item In the case of all identical roots, there are trivially $p$ choices.
            \end{enumerate}
            To get the amount of reducible monic degree 3 polynomials then, we sum all these choices:
            $$\frac{p^3 - p^2}{2} + \frac{p(p-1)(p-2)}{6} + p(p-1) + p = \frac{2p^3 + p}{3}$$
            Find the amount of irreducible monic degree 3 polynomials then just comes from the difference with $p^3$, where we get:
            $$p^3 - \frac{2p^3 + p}{3} = \frac{3p^3}{3} - \frac{2p^3 + p}{3} = \frac{p^3 - p}{3}$$
            Which is the desired answer.
        \end{proof}
        \item Conclude that there is a field of order $p^3$.
        \begin{proof}
            Note for every prime $p$, we have that $\frac{p^3 - p}{3} \geq 1$. Thus for every $p$, there is a monic irreducible polynomial $f$ in $\mathbb{F}_p[x]$ with degree 3. \\ \\
            We appeal then to a theorem proved in class, which has that for $K = \sfrac{\mathbb{F}_p[x]}{(f)}$, we have $[K : \mathbb{F}_p] = \text{deg}(f) = 3$. \\ \\
            This is precisely that the dimension of $K$ is $3$ over the field $\mathbb{F}_p$, i.e. every element of $K$ can be written uniquely as a linear combination of basis vectors with $p^3$ choices of coefficients. \\ \\
            From this it follows $|K| \leq p^3$. But each of the $p^3$ different linear combinations is an element of $K$ (as it is a vector space over $\mathbb{F}_p$), so it must be $|K| \geq p^3$. It follows has precisely order $p^3$. 
        \end{proof}
    \end{enumerate}
\end{ex}

\begin{ex}{3}
    Let $F$ and $K$ be fields and suppose that $K$ is an extension of $F$. Let $\alpha \in K$ be such that $\alpha^2$ is algebraic over $F$. Prove that $\alpha$ is algebraic over $F$.
    \begin{proof}
        Assume $\alpha^2$ is algebraic over $F$, then there exists some polynomial $f \in F[x]$ such that $f(\alpha^2) = 0$. Denote then $f = \sum_{i = 0}^N a_ix^i$, as $f(\alpha^2) = 0$, we get the following:
        \begin{equation}
            \begin{aligned}
                a_N(\alpha^2)^Nx^N + a_{N-1}(\alpha^2)^{N-1}x^{N-1} + ... + a_0 = 0 \\
                a_N\alpha^{2N} + a_{N-1}\alpha^{2N - 2} + ... + a_0 = 0
            \end{aligned}
        \end{equation}
        Define then the polynomial $f_2 = \sum_{i = 1}^N a_ix^{2i}$. Clearly, $f_2 \in F[x]$, as its coefficients are just the coefficients of $f$, which are from $F$. \\ \\
        Moreover, (4) shows that $f_2$ has $\alpha$ as a root, so we thus have that $\alpha$ is algebraic over $F$.
    \end{proof}
\end{ex}

\begin{ex}{4}
    Let $F$ and $K$ be fields and suppose that $K$ is an extension of $F$. An \textit{automorphism} of $K$ over $F$ is an isomorphism $\phi : K \rightarrow K$ such that $\phi(a) = a$ for every $a \in F$. Let $\text{Aut}_F(K)$ be the set of automorphisms of $K$ over $F$.
    \begin{enumerate}[label=(\alph*)]
        \item Prove that $\text{Aut}_F(K)$ is a group under the composition of functions.
        \begin{proof}
            We need to verify the group axioms. For this, let $\phi$ and $\phi$ be automorphisms of $K$ over $F$.
            \begin{enumerate}[label=(\roman*)]
                \item \textit{(Closure under composition)}
                We want that $\phi \circ \psi \in \text{Aut}_F(K)$. For this, recall that the composition of bijections is a bijection. \\ \\
                Moreover, we quickly verify $\phi \circ \psi$ is a homomorphism, as for any elements $a, b \in K$ we have
                $$\phi \circ \psi(a + b) = \phi(\psi(a + b)) = \phi(\psi(a) + \psi(b)) = \phi(\psi(a)) + \phi(\psi(b)) = \phi \circ \psi(a) + \phi \circ \psi(b)$$
                as well as
                $$\phi \circ \psi(ab) = \phi(\psi(ab)) = \phi(\psi(a)\psi(b)) = \phi(\psi(a))\phi(\psi(b)) = \phi \circ \psi(a)\phi \circ \psi(b),$$
                and so $\phi \circ \psi$ is an isomorphism. We just want to check $\phi \circ \psi$ fixes $F$ then, which we see as for $c \in F$ we have
                $$\phi\circ\psi(c) = \phi(\psi(c)) = \phi(c) = c,$$
                and so we have closure under the composition of functions.
                \item \textit{(Existence of an Identity)} We want to verify there is an identity element. \\ \\ We purport this should be the trivial automorphism given by the identity function, i.e. the $\mathbbm{1} : K \rightarrow K$ for which $\phi(a) = a$ for all $a \in K$. \\ \\
                It is trivial that this function is in $\text{Aut}_F(K)$, as the identity is clearly an isomorphism and it of course fixes $F$. \\ \\
                Moreover, it is clear that composition with the identity function just returns the original function, and so the identity function is our identity element.
                \item \textit{(Associativity of Composition)} We just recall the composition of functions in general is associative.
                \item \textit{(Existence of Inverses)} Clearly, $\phi$ admits a bijective inverse $\phi^{-1}$ as it is a bijection. We need to verify then that $\phi^{-1} \in \text{Aut}_F(K)$. \\ \\
                We want first to show then that $\phi^{-1}$ is a homomorphism. For this, we define $c = \phi^{-1}(a + b)$. It follows $\phi(c) = a + b$, but as $\phi$ is a surjection $K \rightarrow K$, we have that $a = \phi(d), b = \phi(e)$ given $d, e \in K$. \\ \\
                Thus $\phi(c) = \phi(d) + \phi (e) = \phi(d + e)$. Using that $\phi$ is an injection, then we have $c = d + e$, but also $d = \phi^{-1}(a)$ and $e = \phi^{-1}(b)$. \\ \\
                Combining these facts then has $\phi^{-1}(a + b) = \phi^{-1}(a) + \phi^{-1}(b)$. Essentially the exact same argument proceeds for multiplication, so we conclude $\phi^{-1}$ is an automorphism of $K$.  \\ \\
                Moreover, as $\phi(c) = c$ for $c \in F$, applying $\phi^{-1}$ to both sides gets $c = \phi^{-1}(c)$, so $\phi^{-1}$ still fixes $F$. Thus $\phi^{-1} \in \text{Aut}_F(K)$.
            \end{enumerate}
            It follows $\text{Aut}_F(K)$ is a group under the composition of functions.
        \end{proof}
        \item If $\alpha \in K$ is algebraic over $F$, prove that $\phi(\alpha)$ is also algebraic over $F$ for every $\phi \in \text{Aut}_F(K)$.
        \begin{proof}
            Let $\alpha \in K$ such that $\alpha$ is algebraic over $F$, that is there is a polynomial $f = \sum_{i = 0}^N a_ix^i \in F[x]$ such that $f(\alpha) = 0$. Note then
            \begin{equation}
                \phi \Big (\sum_{i = 0}^N a_i\alpha^i \Big ) = \phi(0) = 0
            \end{equation}
            as group homomorphisms send the additive identity to the additive identity. Using that $a_i \in F$ for all $i$ then and that $\phi$ fixes elements in $F$, we get
            \begin{equation}
                0 = \phi\Big (\sum_{i = 0}^N a_i\alpha^i \Big ) = \sum_{i = 0}^N \phi(a_i\alpha^i) = \sum_{i = 0}^N \phi(a_i)\phi(\alpha)^i = \sum_{i = 0}^N a_i\phi(\alpha)^i = f(\phi(\alpha)),
            \end{equation}
            which shows that $\phi(a)$ is also a root of $f \in F[x]$, and so it algebraic over $F$. 
        \end{proof}
        \item Find $\text{Aut}_\mathbb{Q}(\mathbb{Q}(i)), \text{Aut}_\mathbb{Q}(\mathbb{Q}(\sqrt[3]{2}))$ and $\text{Aut}_\mathbb{R}(\mathbb{C})$.
        \begin{proof}
            In Homework $\#2$, we showed the only possible \textit{homomorphisms} $\mathbb{Q}(i) \rightarrow \mathbb{Q}(i)$ are the automorphisms given by the identity and conjugation mappings. \\ \\
            Clearly, both of these mappings fix $\mathbb{Q}$, so $\text{Aut}_\mathbb{Q}(\mathbb{Q}(i))$ is just the group consisting of the identity mapping and conjugation mapping, i.e. $\cong \sfrac{\mathbb{Z}}{2\mathbb{Z}}$. \\ \\
            For finding $\text{Aut}_\mathbb{Q}(\mathbb{Q}(\sqrt[3]{2}))$, we let $\phi$ be an automorphism of $\mathbb{Q}(\sqrt[3]{2})$ over $\mathbb{Q}$. We want to determine the value $\phi(\sqrt[3]{2})$, which can done in the following way:
            \begin{equation}
                \begin{aligned}
                    \phi(\sqrt[3]{2}^3 - 2) = \phi(0) = 0 
                    \rightarrow \phi(\sqrt[3]{2}^3) - \phi(2) = 0 \\
                    \rightarrow \phi(\sqrt[3]{2})^3 - 2 = 0 
                    \rightarrow \phi(\sqrt[3]{2})^3 = 2 \\
                    \rightarrow \phi(\sqrt[3]{2}) = \sqrt[3]{2}
                \end{aligned}
            \end{equation}
            This implies then that $\phi$ must be the identity mapping, as for $a + b\sqrt[3]{2}$ we get using the established properties of $\phi$ that
            \begin{equation}
                \phi(a + b\sqrt[3]{2}) = \phi(a) + \phi(b)\phi(\sqrt[3]{2}) = a + b\sqrt[3]{2}
            \end{equation}
            So $\text{Aut}_\mathbb{Q}(\mathbb{Q}(\sqrt[3]{2}))$ just consists of the identity mapping, i.e. it is isomorphic to the trivial group. \\ \\
            For $\text{Aut}_\mathbb{R}(\mathbb{C})$, we do the exact same work as the previous example, except with $i^2 + 1$. Doing this shows that $\phi(i)$ must be a root of $x^2 + 1$, i.e. it is either $i$ or $-i$. \\ \\
            This of course clearly shows any given automorphism of $\mathbb{C}$ over $\mathbb{R}$ is either the identity mapping or the conjugation mapping, as seen in (8).\\ \\
            Thus $\text{Aut}_{\mathbb{R}}(\mathbb{C})$ is the group of the identity and conjugation map, i.e. $\cong \sfrac{\mathbb{Z}}{2\mathbb{Z}}$.
        \end{proof}
    \end{enumerate}
\end{ex}

\end{document}

