\documentclass[12pt]{article}

\usepackage[margin=1in]{geometry} 
\usepackage{amsmath,amsthm,amssymb,enumitem,bbm}

\newcommand{\N}{\mathbb{N}}
\newcommand{\Z}{\mathbb{Z}}
\newcommand{\R}{\mathbb{R}}
\newcommand{\Rd}{\mathbb{R}^{d}}
\newcommand{\exr}{[-\infty, \infty]}

\newenvironment{ex}[2][Exercise]{\begin{trivlist}
\item[\hskip \labelsep {\bfseries #1}\hskip \labelsep {\bfseries #2.}]}{\end{trivlist}}

\newenvironment{sol}[1][Solution]{\begin{trivlist}
\item[\hskip \labelsep {\bfseries #1:}]}{\end{trivlist}}

\newenvironment{theorem}[2][Theorem]{\begin{trivlist}
\item[\hskip \labelsep {\bfseries #1}\hskip \labelsep {\bfseries #2.}]}{\end{trivlist}}
    
\newenvironment{lemma}[2][Lemma]{\begin{trivlist}
\item[\hskip \labelsep {\bfseries #1}\hskip \labelsep {\bfseries #2.}]}{\end{trivlist}}
    
\newenvironment{definition}[2][Definition]{\begin{trivlist}
\item[\hskip \labelsep {\bfseries #1}\hskip \labelsep {\bfseries #2.}]}{\end{trivlist}}
    
\newenvironment{example}[1][Example]{\begin{trivlist}
\item[\hskip \labelsep {\bfseries #1:}]}{\end{trivlist}}
\begin{document}
\noindent David Owen Horace Cutler \hfill {\Large Math 146: Homework 2} \hfill \today

\begin{ex}{1}
    Let $F = \mathbb{Q}(i)$. Consider a non-zero ring homomorphism $\phi : F \rightarrow F$. 
    \begin{enumerate}[label=(\alph*)]
        \item Prove that $\phi$ must be an isomorphism.
        \begin{proof}
            Let $0$ and $1$ denote the additive identity and multiplicative identity in $F$ respectively. \\ \\
            We will show injectivity first. For this, we want to show that $\text{ker}(\phi)$ is trivial, i.e. $\text{ker}(\phi) = \{0\}$. For this, we assume for the sake of contradiction that there is some $a \neq 0$ such that $\phi(a) = 0$. \\ \\
            Note as $F$ is a field, it is an integral domain, as if we have $ab = 0$ with $a, b \neq 0$, we get a contradiction as then they both admits inverses, which has i.e. $a^{-1}ab = a^{-1}(0) \rightarrow b = 0$. \\ \\
            As $F$ is an integral domain and $\phi$ is non-zero then, we have $\phi(1) = 1$, as proven in class. Moreover, as $a \neq 0$, we have that $a^{-1}$ exists. Stitching these facts together yields the following:
            \begin{equation} 1 = \phi(1) = \phi(aa^{-1}) = \phi(a)\phi(a^{-1}) = (0)\phi(a^{-1}) = 0 \end{equation}
            But $F$ is a field, so it should be $1 \neq 0$. Thus (1) yields a contradiction, and so $\text{ker}(\phi)$ is trivial, which furthermore has that $\phi$ is an injection. \\ \\
            (Note by convention it is that $1 \neq 0$ in fields, however, if we do not adopt this convention, we just note that this forces $F$ to be the trivial ring, in which any function $F \rightarrow F$ is a bijection). \\ \\
            To show surjectivity, we need to use some aspects of the structure of $\mathbb{Q}(i)$ (as unlike the injectivity proof, surjectivity need not hold in arbitrary homomorphisms between fields). \\ \\
            In part (b), I prove that at least $\phi$ is onto the rationals as they lay in $\mathbb{Q}(i)$, in particular we have $\phi(a) = a$ for $a \in \mathbb{Q}$. Note then the following as $i^2 = -1$:
            \begin{equation}
                -1 = -\phi(1) = \phi(-1) = \phi(i^2) = \phi(i)^2
            \end{equation}
            Thus $\phi(i) = i \text{ or } -i$. In particular then, for $b \in \mathbb{Q}$, we have that either:
            \begin{equation}
                \phi(bi) = \phi(b)\phi(i) = bi \text{ or } -bi
            \end{equation}
            One observes this has that $bi$ is either mapped to by $bi$ or by $-bi$, depending on how $\phi$ handles $i$. Moreover, in the later case, $\phi(-bi) = -\phi(bi) = -(-bi) = bi$. Thus:
            \begin{equation}
                a + bi = \phi(a) + \phi(bi) \text{ or }\phi(a) + \phi(-bi) = \phi(a + bi) \text{ or } \phi(a - bi)
            \end{equation}
            But this of course has that $\phi$ is surjective, and so $\phi$ is an isomorphism.
        \end{proof}
        \item Prove that $\phi(a) = a$ for $a \in \mathbb{Q}$.
        \begin{proof}
            Let some $a \in \mathbb{Q}$. We express $a = \frac{p}{q}$, where $p \in \mathbb{Z}, q \in \mathbb{N}$. We moreover assume $a \neq 0$, as we already noted $\phi(0) = 0$. \\ \\
            We consider then two cases, the first being $p > 0$. Here, we get the following argument utilizing the fact that $\phi$ is a homomorphism:
            \begin{equation}
                \begin{aligned}
                    \phi \Big(\frac{p}{q}\Big) = \phi \Big (p \frac{1}{q} \Big) = \phi \Big (\sum_{i = 1}^p \frac{1}{q} \Big) = \sum_{i = 1}^p \phi \Big (\frac{1}{q} \Big) = p\phi \Big (\frac{1}{q} \Big ) 
                \end{aligned}
            \end{equation}
            If instead $p < 0$, we make a slight modification:
            \begin{equation}
                \begin{aligned}
                    \phi \Big(\frac{p}{q}\Big) = \phi \Big (p \frac{1}{q} \Big) = \phi \Big (\sum_{i = 1}^{|p|} \frac{-1}{q} \Big) = \sum_{i = 1}^{|p|} \phi \Big ( \frac{-1}{q} \Big ) = \sum_{i = 1}^{|p|} -\phi \Big ( \frac{1}{q} \Big ) = p\phi \Big (\frac{1}{q} \Big)
                \end{aligned}
            \end{equation}
            So regardless $\phi(\frac{p}{q})= q\phi(\frac{1}{q})$. An obvious corollary of this is that for a given $q \in \mathbb{N}$, we have $1 = \phi(1) = q\phi(\frac{1}{q})$. Thus $\phi(\frac{1}{q}) = \frac{1}{q}$. Putting this together with (5) and (6) thus gets $\phi(\frac{p}{q}) = \frac{p}{q}$ in general.
        \end{proof}
        \item Suppose $\alpha \in F$ is such that $\phi(\alpha) = i$. Find a complete list of possibilities for $\alpha$. 
        \begin{proof}
            We want to find all possibilities for $\alpha$. For this, note we have the following:
            \begin{equation}
                \begin{aligned}
                    \phi(\alpha) = i 
                    \longrightarrow \phi(\alpha)^2 = -1 \\
                    \longrightarrow \phi(\alpha^2) = -1 
                    \longrightarrow \phi(\alpha^2) + 1 = 0 \\
                    \longrightarrow \phi(\alpha^2) + \phi(1) = 0 
                    \longrightarrow \phi(\alpha^2 + 1) = 0
                \end{aligned}
            \end{equation}
            But $\phi$ is injective, so this has $\alpha^2 + 1 = 0$. Thus it must be $\alpha = i$ or $\alpha = -i$. Our earlier work in (b) verifies this, as we have either $i$ or $-i$ maps to $i$ as a corrolary of our surjectivity proof.
        \end{proof}
        \item Determine all non-zero homomorphisms $\phi : F \rightarrow F$.
        \begin{proof}
            In part (b), assuming that $\phi$ is a non-zero homomorphism gets us $\phi(1) = 1$ (as $F$ is a field $\rightarrow$ $F$ is an integral domain), which allowed us to prove that in part (a) a small lemma that either $\phi(i) = i \text{ or } -i$. \\ \\
            It's not hard to show then that the value of $\phi(i)$ determines whether or not $\phi$ is one of two possible automorphisms; in particular, we see $\phi$ is either the identity map or the conjugation map by considering $a, b \in \mathbb{Q}$:
            \begin{equation}
                \begin{aligned}
                    \phi(i) = i \rightarrow \phi(a + bi) = \phi(a) +  \phi(b)\phi(i) = a + bi \\
                    \phi(i) = -i \rightarrow \phi(a + bi) = \phi(a) +  \phi(b)\phi(i) = a - bi
                \end{aligned}
            \end{equation}
            As these are the only options for $\phi(i)$ then, these are the only only possible mappings.
        \end{proof}
    \end{enumerate}
\end{ex}

\begin{ex}{2}
    Let $\mathbb{F}_5 = \mathbb{Z}_5$.
    \begin{enumerate}[label=(\alph*)]
        \item Prove that there is no element $\alpha \in \mathbb{F}_5$ such that $\alpha^2 = [2]_5$. 
        \begin{proof}
            We explicitly verify this:
            \begin{equation}
                [0]_5^2 = [0]_5, [1]_5^2 = [1]_5, [2]_5^2 = [4]_5, [3]_5^2 = [4]_5, [4]_5^2 = [1]_5
            \end{equation}
        So there is no "square root of 2" here.
        \end{proof}
        \item Now let $R = \{[a]_5 + [b]_5x : [a]_5, [b]_5 \in \mathbb{F}_5\}$. Consider the normal operations with the added rule that $x^2 = [2]_5$; show that $R$ is a ring under these operations.
        \begin{proof}
            Let some $f, g \in R$ such that $f = [a_0]_5 + [a_1]_5x$, $g = [b_0]_5 + [b_1]_5x$. Then we have:
            \begin{equation}
                \begin{aligned}
                f \cdot g = \sum_{l = 0}^2 \Big ( \sum_{\substack{i + j = l \\ i, j \in \mathbb{N}}} [a_i]_5[b_j]_5 \Big ) = [a_0]_5[b_0]_5 + [a_1]_5[b_0]_5x+ [a_0]_5[b_1]_5x + [a_1][b_1]x^2 \\
                = [a_0b_0]_5 + [a_1b_0 + a_0b_1]x + [2a_1b_1]_5 = [2a_1b_1 + a_0b_0]_5 + [a_1b_0 + a_0b_1]_5x 
                \end{aligned}
            \end{equation}
            This gives us a useful identity for $f \cdot g$, which will allow us to streamline the process of verifying $R$ to be a ring, which we will do now. \\ \\ Note also the fact that $R$ is clearly closed under these operations given (9) (where closure under addition is trivial).
            \begin{enumerate}
                \item \textit{(Commutativity of Addition)} We use the commutativity of addition in $\mathbb{F}_5$:
                \begin{equation}
                    \begin{aligned}
                        f + g = ([a_0]_5 + [a_1]_5x) + ([b_0]_5 + [b_1]_5x) \\
                        = [a_0 + b_0]_5 + [a_1 + b_1]_5x = [b_0 + a_0]_5 + [b_1 + a_1]_5x \\
                        = ([b_0]_5 + [b_1]_5x) + ([a_0]_5 + [a_1]_5x) = g + f
                    \end{aligned}
                \end{equation}
                \item \textit{(Associativity of Addition)} Consider in addition some $h = [c_0]_5 + [c_1]_5x$, we utilize associativity in $\mathbb{F}_5$:
                \begin{equation}
                    \begin{aligned}
                    (f + g) + h = (([a_0]_5 + [a_1]_5x) + ([b_0]_5 + [b_1]_5x)) + ([c_0]_5 + [c_1]_5x) \\
                    = ([a_0 + b_0]_5 + [a_1 + b_1]_5x) + ([c_0]_5 + [c_1]_5x) \\ = 
                    ([(a_0 + b_0) + c_0]_5 + [(a_1 + b_1) + c_1]_5x) \\ =
                    ([a_0 + (b_0 + c_0)]_5 + [a_1 + (b_1 + c_1)]_5x)\\ 
                    = ([a_0]_5 + [a_1]_5x) + ([b_0 + c_0]_5 + [b_1 + c_1]_5x) \\ = ([a_0]_5 + [a_1]_5x) + (([b_0]_5 + [b_1]_5x) + ([c_0]_5 + [c_1]_5x)) = f + (g + h)
                    \end{aligned}
                \end{equation}
                \item \textit{(Existence of an Additive Identity)} We claim the additive identity is just the zero polynomial $[0]_5$ (technically $[0]_5 + [0]_5x$). We verify this quickly, needing only one side as our addition is commutative:
                \begin{equation}
                    \begin{aligned}
                        f + ([0]_5 + [0]_5x) = ([a_0]_5 + [a_1]_5x) + ([0]_5 + [0]_5x) \\ =
                        ([a_0 + 0]_5 + [a_1 + 0]_5x) = ([a_0]_5 + [a_1]_5x) = f
                    \end{aligned}
                \end{equation}
                \item \textit{(Existence of Additive Inverses)} We just levee the additive inverses from $\mathbb{F}_5$, again only needing one side as our addition is commutative:
                \begin{equation}
                    \begin{aligned}
                    f + ([-a_0]_5 + [-a_1]_5x) = ([a_0]_5 + [a_1]_5x) + ([-a_0]_5 + [-a_1]_5x) \\
                    = [a_0 - a_0]_5 + [a_1 - a_1]_5x = [0]_5
                    \end{aligned}
                \end{equation}
                \item \textit{(Associativity of Multiplication)} We use a couple of facts, including distributivity in $\mathbb{F}_5$, as well as commutativity and associaivity of multiplication:
                \begin{equation}
                    \begin{aligned}
                        (f \cdot g) \cdot h = ([2a_1b_1 + a_0b_0]_5 + [a_1b_0 + a_0b_1]x) \cdot ([c_0]_5 + [c_1]_5x) \\
                        = [2a_1b_1c_0 + 2a_1b_1c_0 + 2a_0b_1c_1 + a_0b_0c_0] + [2a_1b_1c_1 + a_0b_0c_1 + a_1b_0c_0 + a_0b_1c_0]x \\
                        = ([a_0]_5 + [a_1]_5x) \cdot ([2b_1c_1 + b_0c_0]_5 + [b_1c_0 + b_0c_1]_5x) = f \cdot (g \cdot h)
                    \end{aligned}
                \end{equation}
                \item \textit{(Existence of a Multiplicative Identity)} We again quickly verify the multiplicative identity is just the polynomial $[1]_5$, needing only one direction as we later prove the multiplication is commutative:
                \begin{equation}
                    \begin{aligned}
                        f \cdot [1]_5 = ([a_0]_5 + [a_1]_5x) \cdot [1]_5 = \\
                        ([a_0(1)]_5 + [a_1(1)]_5x) = ([a_0]_5 + [a_1]_5x) = f
                    \end{aligned}
                \end{equation}
                \item \textit{(Distributivity of Multiplication over Addition)} We use distributivity and commutativity largely to get the following:
                \begin{equation}
                    \begin{aligned}
                         f \cdot (g + h) = ([a_0]_5 + [a_1]_5x) \cdot ([b_0 + c_0]_5 + [b_1 + c_1]_5x) \\
                         = [a_0b_0 + a_0c_0]_5 + [a_0b_1 + a_0c_1]_5x + [a_1b_0 + a_1c_0]x + [2a_1b_1 + 2a_1c_1] \\
                         = ([2a_1b_1 + a_0b_0]_5 + [a_1b_0 + a_0b_1]_5x) + ([2a_1c_1 + a_0c_0]_5 + [a_1c_0 + a_0c_1]x) \\
                         = (f \cdot g) + (f \cdot h)
                    \end{aligned}
                \end{equation}
            \end{enumerate}
            These facts taken together verify $R$ is a ring.
            \end{proof}
            \item Prove that $([a]_5 + [b]_5x)([a]_5 - [b]_5x) = [a^2 - 2b^2] \neq [0]_5$ unless $[a]_5 = [b]_5 = [0]_5$.
            \begin{proof}
                Obviously, this expression is zero when $a$ and $b$ are such. Thus we just need to prove $([a]_5 + [b]_5x)([a]_5 - [b]_5x) = [a^2 - 2b^2] = [0]_5$ implies $[a]_5 = [b]_5 = [0]_5$. \\ \\ For this, we consider the following argument, assuming $b \neq 0$ (so that it has an inverse).
                \begin{equation}
                    \begin{aligned}
                        [a^2 - 2b^2]_5 = [0]_5 \rightarrow [a^2]_5 = [2b^2]_5 \\
                        \rightarrow [a]^2_5 = [2]_5[b]_5^2 \rightarrow [a]^2[b]_5^{-2} = [2]_5 \\
                        \rightarrow ([a][b]^{-1}_5)^2 = [2]_5
                    \end{aligned}
                \end{equation}
                But then this is a contradiction as $[a][b]_5^{-1} \in \mathbb{F}_5$ and we showed there is no element that has $[2]_5$ as its square. \\ \\
                So it must be $[b]_5 = 0$. It follows then $[a^2 - 2b^2] = [a^2] = [a]^2 = [0]_5$, but we trivially verify the only element in $\mathbb{F}_5$ with its square as $[0]_5$ is just $[0]_5$. \\ \\
                Thus here $[a]_5 = [b]_5 = 0$ as desired.
            \end{proof}
            \item Prove that $R$ is a field. 
            \begin{proof}
                We need to verify now, in addition, commutativity of multiplication and the existence of multiplicative inverses for nonzero elements. Note we already have $1 \neq 0$ here.
                \begin{enumerate}
                    \item \textit{(Commutativity of Multiplication)} We just use commutativity in $\mathbb{F}_5$:
                    \begin{equation}
                        \begin{aligned}
                            f \cdot g = ([a_0]_5 + [a_1]_5x) \cdot ([b_0]_5 + [b_1]_5) \\
                            = [2a_1b_1 + a_0b_0]_5 + [a_1b_0 + a_0b_1]_5x = [2b_1a_1 + b_0a_0]_5 + [b_1a_0 + b_0a_1]_5x \\
                            = g \cdot f
                        \end{aligned}
                    \end{equation}
                    \item \textit{(Existence of Multiplicative Inverses)} We claim that for nonzero polynomial $[a]_5 + [b]_5x$, its inverse is $[a]_5[a^2 - 2b^2]_5^{-1} + [-b]_5[a^2 - 2b^2]_5^{-1}x$. \\ \\
                    Note this is where we are using part (c), as we know as long as our polynomial isn't zero (i.e. $[a]_5 \neq [0]_5$ or $[b]_5 \neq [0]_5$) that $[a^2 - 2b^2]_5$ is nonzero, and thus it has an inverse. 
                    \begin{equation}
                        \begin{aligned}
                            f \cdot ([a]_5[a^2 - 2b^2]_5^{-1} + [-b]_5[a^2 - 2b^2]_5^{-1}x) = \\
                            ([a]_5 + [b_5]x) \cdot ([a]_5[a^2 - 2b^2]_5^{-1} + [-b]_5[a^2 - 2b^2]_5^{-1}x) \\
                            = [a^2][a^2 - 2b^2]_5^{-1} + [-2b^2][a^2 - 2b^2]_5^{-1} = \\
                            ([a^2 - 2b^2]_5)[a^2 - 2b^2]_5^{-1} = [1]_5
                        \end{aligned}
                    \end{equation}
                \end{enumerate}
                And so $R$ is a field.
            \end{proof}
    \end{enumerate}
\end{ex}

\begin{ex}{3}
    Let $R$ be a commutative ring and let $I, J \subseteq R$ be ideals. We define a set $I + J$ as the set of all possibles sums of elements in $I$ and $J$, i.e. $I + J = \{i + j : i \in I, j \in J\}$.
    \begin{enumerate}[label=(\alph*)]
        \item Prove that $I + J$ is an ideal.
        \begin{proof}
            We first show that $I + J$ is an additive subgroup of $R$. For this, first note $I + J$ is nonempty, as all ideals include the additive identity (and so the sum $0 + 0 = 0 \in I + J$). \\ \\
            Let then $(i_1 + j_1), (i_2 + j_2) \in I + J$. To show that $I + J$ is an additive subgroup, we want that the following holds:
            \begin{equation}
                (i_1 + j_1) - (i_2 + j_2) \in I + J
            \end{equation}
            Note then that $-(i_2 + j_2) = -i_2 -j_2$. Thus we reorganize the expression to get $(i_1 - i_2) + (j_1 - j_2)$. \\ \\
            But this of course must be in $I + J$, as we have $i_1 - i_2 \in I$ and $j_1 - j_2 \in J$, given that they are ideals. \\ \\
            That $I + J$ is closed under multiplication with outside elements is even simpler, we just note for all $r \in R$:
            \begin{equation}
                (i_1 + j_1)r = i_1r + j_1r
            \end{equation}
            But then again, as $I$ and $J$ are ideals, the terms on the right are in $I$ and $J$ respectively, so the left term is in $I + J$ (as $R$ is commutative, we only need to check one side of multiplication). \\ \\
            It follows $I + J$ is an ideal.
        \end{proof}
        \item Suppose $K \subseteq R$ is any ideal containing both $I$ and $J$. Show that $I + J \subseteq K$.
        \begin{proof}
            Consider some element $i_1 + j_1 \in I + J$. We know $i_1 \in I \subseteq K$ and $j_1 \in J \subseteq K$. Then, as $K$ is an ideal (and thus an additive subgroup), we must have $i_1 + j_1 \in K$, but this proves it.
        \end{proof}
        \item Prove that $I \cap J$ is an ideal.
        \begin{proof}
            First note $I \cap J$ is nonempty as all ideals must contain the additive identity (i.e. it contains the additive identity). Consider then some $k_1, k_2 \in I \cap J$. \\ \\
            Of course then, $k_1, k_2 \in I$ and $k_1, k_2 \in J$. Thus $k_1 - k_2 \in I$ and $k_1 - k_2 \in J$ (using additive subgroup properties), so we have $k_1 - k_2 \in I \cap J$. \\ \\
            Thus $I \cap J$ is an additive subgroup. Similarly, we have for all $r \in R$ that $k_1r \in I$ given $k_1 \in I$ and and $k_1r \in J$ given $k_1 \in J$. \\ \\
            Thus $k_1r \in I \cap J$, so $I \cap J$ is an ideal.
        \end{proof}
    \end{enumerate}
\end{ex}

\end{document}